% Options for packages loaded elsewhere
\PassOptionsToPackage{unicode}{hyperref}
\PassOptionsToPackage{hyphens}{url}
%
\documentclass[
  ignorenonframetext,
]{beamer}
\usepackage{pgfpages}
\setbeamertemplate{caption}[numbered]
\setbeamertemplate{caption label separator}{: }
\setbeamercolor{caption name}{fg=normal text.fg}
\beamertemplatenavigationsymbolsempty
% Prevent slide breaks in the middle of a paragraph
\widowpenalties 1 10000
\raggedbottom
\setbeamertemplate{part page}{
  \centering
  \begin{beamercolorbox}[sep=16pt,center]{part title}
    \usebeamerfont{part title}\insertpart\par
  \end{beamercolorbox}
}
\setbeamertemplate{section page}{
  \centering
  \begin{beamercolorbox}[sep=12pt,center]{part title}
    \usebeamerfont{section title}\insertsection\par
  \end{beamercolorbox}
}
\setbeamertemplate{subsection page}{
  \centering
  \begin{beamercolorbox}[sep=8pt,center]{part title}
    \usebeamerfont{subsection title}\insertsubsection\par
  \end{beamercolorbox}
}
\AtBeginPart{
  \frame{\partpage}
}
\AtBeginSection{
  \ifbibliography
  \else
    \frame{\sectionpage}
  \fi
}
\AtBeginSubsection{
  \frame{\subsectionpage}
}
\usepackage{lmodern}
\usepackage{amssymb,amsmath}
\usepackage{ifxetex,ifluatex}
\ifnum 0\ifxetex 1\fi\ifluatex 1\fi=0 % if pdftex
  \usepackage[T1]{fontenc}
  \usepackage[utf8]{inputenc}
  \usepackage{textcomp} % provide euro and other symbols
\else % if luatex or xetex
  \usepackage{unicode-math}
  \defaultfontfeatures{Scale=MatchLowercase}
  \defaultfontfeatures[\rmfamily]{Ligatures=TeX,Scale=1}
\fi
\usetheme[]{CambridgeUS}
\usecolortheme{sidebartab}
% Use upquote if available, for straight quotes in verbatim environments
\IfFileExists{upquote.sty}{\usepackage{upquote}}{}
\IfFileExists{microtype.sty}{% use microtype if available
  \usepackage[]{microtype}
  \UseMicrotypeSet[protrusion]{basicmath} % disable protrusion for tt fonts
}{}
\makeatletter
\@ifundefined{KOMAClassName}{% if non-KOMA class
  \IfFileExists{parskip.sty}{%
    \usepackage{parskip}
  }{% else
    \setlength{\parindent}{0pt}
    \setlength{\parskip}{6pt plus 2pt minus 1pt}}
}{% if KOMA class
  \KOMAoptions{parskip=half}}
\makeatother
\usepackage{xcolor}
\IfFileExists{xurl.sty}{\usepackage{xurl}}{} % add URL line breaks if available
\IfFileExists{bookmark.sty}{\usepackage{bookmark}}{\usepackage{hyperref}}
\hypersetup{
  pdftitle={Importing data from different formats},
  hidelinks,
  pdfcreator={LaTeX via pandoc}}
\urlstyle{same} % disable monospaced font for URLs
\newif\ifbibliography
\usepackage{color}
\usepackage{fancyvrb}
\newcommand{\VerbBar}{|}
\newcommand{\VERB}{\Verb[commandchars=\\\{\}]}
\DefineVerbatimEnvironment{Highlighting}{Verbatim}{commandchars=\\\{\}}
% Add ',fontsize=\small' for more characters per line
\usepackage{framed}
\definecolor{shadecolor}{RGB}{248,248,248}
\newenvironment{Shaded}{\begin{snugshade}}{\end{snugshade}}
\newcommand{\AlertTok}[1]{\textcolor[rgb]{0.94,0.16,0.16}{#1}}
\newcommand{\AnnotationTok}[1]{\textcolor[rgb]{0.56,0.35,0.01}{\textbf{\textit{#1}}}}
\newcommand{\AttributeTok}[1]{\textcolor[rgb]{0.77,0.63,0.00}{#1}}
\newcommand{\BaseNTok}[1]{\textcolor[rgb]{0.00,0.00,0.81}{#1}}
\newcommand{\BuiltInTok}[1]{#1}
\newcommand{\CharTok}[1]{\textcolor[rgb]{0.31,0.60,0.02}{#1}}
\newcommand{\CommentTok}[1]{\textcolor[rgb]{0.56,0.35,0.01}{\textit{#1}}}
\newcommand{\CommentVarTok}[1]{\textcolor[rgb]{0.56,0.35,0.01}{\textbf{\textit{#1}}}}
\newcommand{\ConstantTok}[1]{\textcolor[rgb]{0.00,0.00,0.00}{#1}}
\newcommand{\ControlFlowTok}[1]{\textcolor[rgb]{0.13,0.29,0.53}{\textbf{#1}}}
\newcommand{\DataTypeTok}[1]{\textcolor[rgb]{0.13,0.29,0.53}{#1}}
\newcommand{\DecValTok}[1]{\textcolor[rgb]{0.00,0.00,0.81}{#1}}
\newcommand{\DocumentationTok}[1]{\textcolor[rgb]{0.56,0.35,0.01}{\textbf{\textit{#1}}}}
\newcommand{\ErrorTok}[1]{\textcolor[rgb]{0.64,0.00,0.00}{\textbf{#1}}}
\newcommand{\ExtensionTok}[1]{#1}
\newcommand{\FloatTok}[1]{\textcolor[rgb]{0.00,0.00,0.81}{#1}}
\newcommand{\FunctionTok}[1]{\textcolor[rgb]{0.00,0.00,0.00}{#1}}
\newcommand{\ImportTok}[1]{#1}
\newcommand{\InformationTok}[1]{\textcolor[rgb]{0.56,0.35,0.01}{\textbf{\textit{#1}}}}
\newcommand{\KeywordTok}[1]{\textcolor[rgb]{0.13,0.29,0.53}{\textbf{#1}}}
\newcommand{\NormalTok}[1]{#1}
\newcommand{\OperatorTok}[1]{\textcolor[rgb]{0.81,0.36,0.00}{\textbf{#1}}}
\newcommand{\OtherTok}[1]{\textcolor[rgb]{0.56,0.35,0.01}{#1}}
\newcommand{\PreprocessorTok}[1]{\textcolor[rgb]{0.56,0.35,0.01}{\textit{#1}}}
\newcommand{\RegionMarkerTok}[1]{#1}
\newcommand{\SpecialCharTok}[1]{\textcolor[rgb]{0.00,0.00,0.00}{#1}}
\newcommand{\SpecialStringTok}[1]{\textcolor[rgb]{0.31,0.60,0.02}{#1}}
\newcommand{\StringTok}[1]{\textcolor[rgb]{0.31,0.60,0.02}{#1}}
\newcommand{\VariableTok}[1]{\textcolor[rgb]{0.00,0.00,0.00}{#1}}
\newcommand{\VerbatimStringTok}[1]{\textcolor[rgb]{0.31,0.60,0.02}{#1}}
\newcommand{\WarningTok}[1]{\textcolor[rgb]{0.56,0.35,0.01}{\textbf{\textit{#1}}}}
\usepackage{longtable,booktabs}
\usepackage{caption}
% Make caption package work with longtable
\makeatletter
\def\fnum@table{\tablename~\thetable}
\makeatother
\usepackage{graphicx,grffile}
\makeatletter
\def\maxwidth{\ifdim\Gin@nat@width>\linewidth\linewidth\else\Gin@nat@width\fi}
\def\maxheight{\ifdim\Gin@nat@height>\textheight\textheight\else\Gin@nat@height\fi}
\makeatother
% Scale images if necessary, so that they will not overflow the page
% margins by default, and it is still possible to overwrite the defaults
% using explicit options in \includegraphics[width, height, ...]{}
\setkeys{Gin}{width=\maxwidth,height=\maxheight,keepaspectratio}
% Set default figure placement to htbp
\makeatletter
\def\fps@figure{htbp}
\makeatother
\setlength{\emergencystretch}{3em} % prevent overfull lines
\providecommand{\tightlist}{%
  \setlength{\itemsep}{0pt}\setlength{\parskip}{0pt}}
\setcounter{secnumdepth}{-\maxdimen} % remove section numbering

\title{Importing data from different formats}
\author{}
\date{\vspace{-2.5em}}

\begin{document}
\frame{\titlepage}

\begin{frame}{RECAP: Data Types / Classes}
\protect\hypertarget{recap-data-types-classes}{}

\begin{longtable}[]{@{}ll@{}}
\toprule
Data Types & Stores\tabularnewline
\midrule
\endhead
real & floating point numbers\tabularnewline
integer & integers\tabularnewline
complex & Complex numbers\tabularnewline
factor & categorical data\tabularnewline
character & strings\tabularnewline
logical & TRUE or FALSe\tabularnewline
NA & Missing\tabularnewline
NULL & Empty\tabularnewline
Function & Function type\tabularnewline
-------------------- & ------------------------------\tabularnewline
\bottomrule
\end{longtable}

\end{frame}

\begin{frame}[fragile]{Vector}
\protect\hypertarget{vector}{}

\begin{itemize}
\tightlist
\item
  A vector can only contain objects of the same class
\end{itemize}

\begin{Shaded}
\begin{Highlighting}[]
\NormalTok{a <-}\StringTok{ }\KeywordTok{c}\NormalTok{(}\DecValTok{1}\NormalTok{,}\DecValTok{2}\NormalTok{,}\FloatTok{5.3}\NormalTok{,}\DecValTok{6}\NormalTok{,}\OperatorTok{-}\DecValTok{2}\NormalTok{,}\DecValTok{4}\NormalTok{) }\CommentTok{# numeric vector}
\NormalTok{b <-}\StringTok{ }\KeywordTok{c}\NormalTok{(}\StringTok{"one"}\NormalTok{,}\StringTok{"two"}\NormalTok{,}\StringTok{"three"}\NormalTok{) }\CommentTok{# character vector}
\NormalTok{c <-}\StringTok{ }\KeywordTok{c}\NormalTok{(}\OtherTok{TRUE}\NormalTok{,}\OtherTok{TRUE}\NormalTok{,}\OtherTok{TRUE}\NormalTok{,}\OtherTok{FALSE}\NormalTok{,}\OtherTok{TRUE}\NormalTok{,}\OtherTok{FALSE}\NormalTok{) }\CommentTok{#logical vector}
\end{Highlighting}
\end{Shaded}

\end{frame}

\begin{frame}[fragile]{Matrices}
\protect\hypertarget{matrices}{}

\begin{itemize}
\tightlist
\item
  All columns in a matrix must have the same class(numeric, character,
  etc.) and the same length. The general format is
\end{itemize}

\begin{Shaded}
\begin{Highlighting}[]
\CommentTok{#mymatrix <- matrix(vector, nrow=r, ncol=c, byrow=FALSE,}
\CommentTok{#dimnames=list(char_vector_rownames, char_vector_colnames)) }
\CommentTok{#byrow=TRUE indicates that the matrix should be filled by rows}
\end{Highlighting}
\end{Shaded}

\end{frame}

\begin{frame}[fragile]{Factors}
\protect\hypertarget{factors}{}

\begin{itemize}
\tightlist
\item
  Used to represent categorical data.
\item
  Can be unordered or ordered. -A factor is like an integer vector where
  each integer has a label.
\end{itemize}

\begin{Shaded}
\begin{Highlighting}[]
\NormalTok{  x <-}\StringTok{ }\KeywordTok{factor}\NormalTok{(}\KeywordTok{c}\NormalTok{(}\StringTok{"yes"}\NormalTok{, }\StringTok{"yes"}\NormalTok{, }\StringTok{"no"}\NormalTok{, }\StringTok{"yes"}\NormalTok{, }\StringTok{"no"}\NormalTok{))}
\NormalTok{  x}
\end{Highlighting}
\end{Shaded}

\begin{verbatim}
## [1] yes yes no  yes no 
## Levels: no yes
\end{verbatim}

\end{frame}

\begin{frame}[fragile]{Missing Values}
\protect\hypertarget{missing-values}{}

\begin{itemize}
\item
  Missing values are represented by the symbol \textbf{NA} (not
  available)
\item
  Impossible values (e.g., dividing by zero) are represented by the
  symbol NaN (not a number)
\item
  Can be unordered or ordered. -A factor is like an integer vector where
  each integer has a label.
\end{itemize}

\begin{Shaded}
\begin{Highlighting}[]
\NormalTok{  x <-}\StringTok{ }\OtherTok{NA}
 \CommentTok{# is.na(x) # returns TRUE of x is missing}
\CommentTok{# mean(x, na.rm=TRUE) # exclude missing in functions}
\CommentTok{# complete.cases() #returns the number of complete cases}
\end{Highlighting}
\end{Shaded}

\end{frame}

\begin{frame}{Data Frames}
\protect\hypertarget{data-frames}{}

\begin{itemize}
\tightlist
\item
  More general than a matrix, has different columns and can have
  different modes (numeric, character, factor, etc.)
\item
  Used to store tabular data
\item
  Can store data of different classes
\item
  \emph{read.table()} or \emph{read.csv()} -- used to load dataframes
\end{itemize}

\end{frame}

\begin{frame}[fragile]{Create Data Frames}
\protect\hypertarget{create-data-frames}{}

\begin{Shaded}
\begin{Highlighting}[]
\KeywordTok{data.frame}\NormalTok{(}\DataTypeTok{foo =} \DecValTok{1}\OperatorTok{:}\DecValTok{4}\NormalTok{, }\DataTypeTok{bar =} \KeywordTok{c}\NormalTok{(T, T, F, F))}
\end{Highlighting}
\end{Shaded}

\begin{verbatim}
##   foo   bar
## 1   1  TRUE
## 2   2  TRUE
## 3   3 FALSE
## 4   4 FALSE
\end{verbatim}

\begin{Shaded}
\begin{Highlighting}[]
\NormalTok{x <-}\StringTok{ }\KeywordTok{c}\NormalTok{(}\DecValTok{1}\NormalTok{, }\DecValTok{2}\NormalTok{,}\DecValTok{3}\NormalTok{,}\DecValTok{4}\NormalTok{,}\DecValTok{5}\NormalTok{,}\DecValTok{6}\NormalTok{,}\DecValTok{7}\NormalTok{,}\DecValTok{8}\NormalTok{,}\DecValTok{9}\NormalTok{)}
\NormalTok{y <-}\StringTok{ }\KeywordTok{c}\NormalTok{(}\StringTok{"a"}\NormalTok{,}\StringTok{"b"}\NormalTok{,}\StringTok{"c"}\NormalTok{,}\StringTok{"d"}\NormalTok{,}\StringTok{"e"}\NormalTok{,}\StringTok{"f"}\NormalTok{,}\StringTok{"g"}\NormalTok{,}\StringTok{"h"}\NormalTok{,}\StringTok{"i"}\NormalTok{)}
\NormalTok{df <-}\StringTok{ }\KeywordTok{data.frame}\NormalTok{(}\DataTypeTok{x=}\NormalTok{x, }\DataTypeTok{y=}\NormalTok{y)}
\end{Highlighting}
\end{Shaded}

\end{frame}

\begin{frame}[fragile]

\begin{Shaded}
\begin{Highlighting}[]
\KeywordTok{print}\NormalTok{(df)}
\end{Highlighting}
\end{Shaded}

\begin{verbatim}
##   x y
## 1 1 a
## 2 2 b
## 3 3 c
## 4 4 d
## 5 5 e
## 6 6 f
## 7 7 g
## 8 8 h
## 9 9 i
\end{verbatim}

\begin{Shaded}
\begin{Highlighting}[]
\KeywordTok{class}\NormalTok{(df)}
\end{Highlighting}
\end{Shaded}

\begin{verbatim}
## [1] "data.frame"
\end{verbatim}

\end{frame}

\begin{frame}{Datasets}
\protect\hypertarget{datasets}{}

\begin{itemize}
\tightlist
\item
  R works with different types of datasets
\item
  Base R functions \emph{read.table} , \emph{read.csv} and
  \emph{read.delim} can read in data stored as text files, delimited by
  \emph{almost anything}
\item
  Data from other stat packages can be read using \emph{foreign
  package?} and \emph{Hmisc package} \newline \emph{read.xlsx(file,
  sheetIndex=1) \#excel files} \newline \emph{read.dta(file)\# stata
  files}
\end{itemize}

\end{frame}

\begin{frame}{.RDA Data}
\protect\hypertarget{rda-data}{}

\begin{itemize}
\tightlist
\item
  R Data type
\item
  Can be created from other data sets -data \textless-
  load(``profit.rda'') -Saving a data frame as an rda

  \begin{itemize}
  \tightlist
  \item
    Save(data.frame, ``dataset.rda'')
  \end{itemize}
\end{itemize}

\end{frame}

\begin{frame}[fragile]{Examples: 1}
\protect\hypertarget{examples-1}{}

\begin{Shaded}
\begin{Highlighting}[]
\CommentTok{# Reading data from SPSS using the package "foreign"}
\KeywordTok{library}\NormalTok{(foreign)}
\NormalTok{data1<-}\KeywordTok{read.spss}\NormalTok{(}\StringTok{"experim.sav"}\NormalTok{, }\DataTypeTok{to.data.frame=}\OtherTok{TRUE}\NormalTok{)}
\end{Highlighting}
\end{Shaded}

\begin{verbatim}
## re-encoding from CP1252
\end{verbatim}

\begin{Shaded}
\begin{Highlighting}[]
\NormalTok{data1}
\end{Highlighting}
\end{Shaded}

\begin{verbatim}
##    id    sex age               group fost1 confid1 depress1 fost2 confid2
## 1   4   male  23 confidence building    50      15       44    48      16
## 2  10   male  21 confidence building    47      14       42    45      15
## 3   9   male  25        maths skills    44      12       40    39      18
## 4   3   male  30        maths skills    47      11       43    42      16
## 5  12   male  45 confidence building    46      16       44    45      16
## 6  11   male  22        maths skills    39      13       43    40      20
## 7   6   male  22 confidence building    32      21       37    33      22
## 8   5   male  26        maths skills    44      17       46    37      20
## 9   8   male  23 confidence building    40      22       37    40      23
## 10 13   male  21        maths skills    47      20       50    45      25
## 11 14   male  23 confidence building    38      28       39    37      27
## 12  1   male  19        maths skills    32      20       44    28      25
## 13 15   male  23        maths skills    39      21       47    35      26
## 14  7   male  19        maths skills    36      24       38    32      28
## 15  2   male  21 confidence building    37      29       50    36      30
## 16 27 female  20        maths skills    41      16       45    40      14
## 17 25 female  24        maths skills    38      14       42    37      14
## 18 19 female  27        maths skills    42      15       49    41      13
## 19 18 female  23 confidence building    44      13       39    39      20
## 20 23 female  22        maths skills    32      22       39    31      18
## 21 21 female  46        maths skills    39      21       44    40      19
## 22 26 female  19 confidence building    42      13       43    38      20
## 23 29 female  22        maths skills    37      28       33    38      22
## 24 17 female  37        maths skills    41      29       39    40      22
## 25 20 female  32 confidence building    43      17       47    36      26
## 26 28 female  30 confidence building    46      20       38    40      28
## 27 22 female  25 confidence building    30      24       45    28      28
## 28 24 female  21 confidence building    33      12       50    29      20
## 29 16 female  45 confidence building    40      22       45    30      35
## 30 30 female  21 confidence building    39      21       34    36      30
##    depress2 fost3 confid3 depress3 exam      mah_1      DepT1gp2      DepT2Gp2
## 1        44    45      14       40   52  0.5699842 not depressed not depressed
## 2        42    44      18       40   55  1.6594031 not depressed not depressed
## 3        40    36      19       38   58  3.5404715 not depressed not depressed
## 4        43    41      20       43   60  2.4542143 not depressed not depressed
## 5        45    43      20       43   58  0.9443036 not depressed     depressed
## 6        42    39      22       38   62  1.6257058 not depressed not depressed
## 7        36    32      23       35   59  4.1744717 not depressed not depressed
## 8        47    32      26       42   70  1.0261059     depressed     depressed
## 9        37    40      26       35   60  1.7053103 not depressed not depressed
## 10       48    46      27       46   70  3.0873214     depressed     depressed
## 11       36    32      29       34   72  2.9140163 not depressed not depressed
## 12       43    23      30       40   82  0.3469978 not depressed not depressed
## 13       47    35      30       47   79  1.5886241     depressed     depressed
## 14       35    30      32       35   80  1.5076582 not depressed not depressed
## 15       47    34      34       45   90 10.2401804     depressed     depressed
## 16       44    38      18       40   56  1.1776467     depressed not depressed
## 17       40    35      19       39   53  1.0564668 not depressed not depressed
## 18       49    40      20       44   59  3.8748910     depressed     depressed
## 19       30    34      22       30   64  2.7101641 not depressed not depressed
## 20       38    32      22       36   63  3.5488594 not depressed not depressed
## 21       44    38      23       44   64  0.5007192 not depressed not depressed
## 22       39    36      23       37   63  1.4739118 not depressed not depressed
## 23       33    36      26       32   67  9.1295758 not depressed not depressed
## 24       40    40      27       40   71  6.2065842 not depressed not depressed
## 25       45    34      28       42   73  1.7190981     depressed     depressed
## 26       30    37      29       29   80  1.5015533 not depressed not depressed
## 27       40    25      30       38   83  1.9240376     depressed not depressed
## 28       48    25      30       50   85  7.5576411     depressed     depressed
## 29       40    25      32       42   78  1.1884922     depressed not depressed
## 30       30    30      32       32   84  6.0455901 not depressed not depressed
##         DepT3gp2
## 1  not depressed
## 2  not depressed
## 3  not depressed
## 4  not depressed
## 5  not depressed
## 6  not depressed
## 7  not depressed
## 8  not depressed
## 9  not depressed
## 10     depressed
## 11 not depressed
## 12 not depressed
## 13     depressed
## 14 not depressed
## 15     depressed
## 16 not depressed
## 17 not depressed
## 18 not depressed
## 19 not depressed
## 20 not depressed
## 21 not depressed
## 22 not depressed
## 23 not depressed
## 24 not depressed
## 25 not depressed
## 26 not depressed
## 27 not depressed
## 28     depressed
## 29 not depressed
## 30 not depressed
\end{verbatim}

\end{frame}

\begin{frame}[fragile]{Examples: 2}
\protect\hypertarget{examples-2}{}

\begin{Shaded}
\begin{Highlighting}[]
\CommentTok{# Reading data from STATA using the package "foreign"}
\NormalTok{data2<-}\KeywordTok{read.dta}\NormalTok{(}\StringTok{"cr4.dta"}\NormalTok{)}
\NormalTok{data2}
\end{Highlighting}
\end{Shaded}

\begin{verbatim}
##     y a order
## 1   4 1     1
## 2   6 1     2
## 3   3 1     3
## 4   3 1     4
## 5   1 1     5
## 6   3 1     6
## 7   2 1     7
## 8   2 1     8
## 9   4 2     1
## 10  5 2     2
## 11  4 2     3
## 12  3 2     4
## 13  2 2     5
## 14  3 2     6
## 15  4 2     7
## 16  3 2     8
## 17  5 3     1
## 18  6 3     2
## 19  5 3     3
## 20  4 3     4
## 21  3 3     5
## 22  4 3     6
## 23  3 3     7
## 24  4 3     8
## 25  3 4     1
## 26  5 4     2
## 27  6 4     3
## 28  5 4     4
## 29  6 4     5
## 30  7 4     6
## 31  8 4     7
## 32 10 4     8
\end{verbatim}

\end{frame}

\begin{frame}[fragile]{Export to csv}
\protect\hypertarget{export-to-csv}{}

\begin{itemize}
\tightlist
\item
  Use \emph{write.csv} to export data frames
\end{itemize}

\begin{verbatim}

write.csv(Your DataFrame,"Path to export the DataFrame")
File Name.csv", row.names = FALSE)
\end{verbatim}

\begin{Shaded}
\begin{Highlighting}[]
\KeywordTok{write.csv}\NormalTok{(data2,}\StringTok{"cr4.csv"}\NormalTok{, }\DataTypeTok{row.names =} \OtherTok{FALSE}\NormalTok{)}
\end{Highlighting}
\end{Shaded}

\end{frame}

\begin{frame}[fragile]{Importing data from excel, csv}
\protect\hypertarget{importing-data-from-excel-csv}{}

-This is the most common format we use -Data files saved in excel can be
saved in the several formats:

\begin{verbatim}
.xlsx #Excel format -
.csv  #comma seperate values
.txt #tab delimited
\end{verbatim}

\begin{itemize}
\item
  data must always first be organized in the right format see- Broman
  and Woo, 2017

  Link:
  \url{https://www.tandfonline.com/doi/full/10.1080/00031305.2017.1375989}
\end{itemize}

\end{frame}

\begin{frame}[fragile]{Example}
\protect\hypertarget{example}{}

\begin{Shaded}
\begin{Highlighting}[]
\KeywordTok{read.csv}\NormalTok{(}\StringTok{"cr4.csv"}\NormalTok{)}
\end{Highlighting}
\end{Shaded}

\begin{verbatim}
##     y a order
## 1   4 1     1
## 2   6 1     2
## 3   3 1     3
## 4   3 1     4
## 5   1 1     5
## 6   3 1     6
## 7   2 1     7
## 8   2 1     8
## 9   4 2     1
## 10  5 2     2
## 11  4 2     3
## 12  3 2     4
## 13  2 2     5
## 14  3 2     6
## 15  4 2     7
## 16  3 2     8
## 17  5 3     1
## 18  6 3     2
## 19  5 3     3
## 20  4 3     4
## 21  3 3     5
## 22  4 3     6
## 23  3 3     7
## 24  4 3     8
## 25  3 4     1
## 26  5 4     2
## 27  6 4     3
## 28  5 4     4
## 29  6 4     5
## 30  7 4     6
## 31  8 4     7
## 32 10 4     8
\end{verbatim}

\end{frame}

\begin{frame}{Paths}
\protect\hypertarget{paths}{}

\begin{itemize}
\item
  Note that when importing your data you must know where the file is
\item
  Helps you tell the computer where to find the data
  \includegraphics{path.png}
\item
  Absolute path
\end{itemize}

\end{frame}

\begin{frame}{Paths}
\protect\hypertarget{paths-1}{}

\begin{itemize}
\tightlist
\item
  Types of paths

  \begin{itemize}
  \tightlist
  \item
    Absolute path -fixed and things must always be in the location
    specified
  \item
    Relative path- allows you to move entire folders and retain folder
    structure
  \end{itemize}
\item
  It is best to create workspace/ projects using Rstudio
\item
  This allows you to move folders and retain folder structure
\end{itemize}

\end{frame}

\begin{frame}[fragile]{Importing using Absolute path}
\protect\hypertarget{importing-using-absolute-path}{}

\begin{Shaded}
\begin{Highlighting}[]
\CommentTok{# reading a txt file}
\CommentTok{# read.table("D:/F-STAR/data/cr4.txt")}
\end{Highlighting}
\end{Shaded}

\begin{Shaded}
\begin{Highlighting}[]
\CommentTok{# reading a excel file}
\KeywordTok{library}\NormalTok{(readxl)}
\KeywordTok{read_xlsx}\NormalTok{(}\StringTok{"cr4.xlsx"}\NormalTok{,}\DataTypeTok{sheet=}\DecValTok{1}\NormalTok{, }\DataTypeTok{col_names =} \OtherTok{TRUE}\NormalTok{)}
\end{Highlighting}
\end{Shaded}

\end{frame}

\begin{frame}[fragile]{Importing using relative path}
\protect\hypertarget{importing-using-relative-path}{}

\begin{itemize}
\tightlist
\item
  Here use either the Rstudio Projects or set working directory
  \emph{setwd} approach
\end{itemize}

\begin{Shaded}
\begin{Highlighting}[]
\CommentTok{# Set working directory}
\CommentTok{# read_xlsx("D:/F-STAR/data/cr4.xlsx",sheet=1)}
\KeywordTok{setwd}\NormalTok{(}\StringTok{"D:/F-STAR"}\NormalTok{)}
\KeywordTok{read_xlsx}\NormalTok{(}\StringTok{"cr4.xlsx"}\NormalTok{,}\DataTypeTok{sheet=}\DecValTok{1}\NormalTok{,}\DataTypeTok{col_names =} \OtherTok{TRUE}\NormalTok{)}
\end{Highlighting}
\end{Shaded}

\end{frame}

\end{document}
